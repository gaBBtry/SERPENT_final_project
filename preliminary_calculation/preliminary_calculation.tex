\section{Calculs préliminaires}

\subsection{Points clés}
\begin{itemize}
  \item Il semble probable que le nombre de fissions annuelles dans l'assemblage soit d'environ \SI{1,5e25}{fissions}, basé sur la puissance thermique estimée.
  \item Les taux de fission attendus pour les éléments fissiles principaux (U-235, Pu-239, Pu-241) sont d'environ \SI{1,1e17}{}, \SI{3,1e17}{} et \SI{6,4e16}{fissions\per\second} respectivement, en utilisant des sections efficaces thermiques de la littérature.
  \item L'ordre de grandeur du flux de neutrons est estimé à \SI{1e17}{\neutron\per\centi\meter\squared\per\second}, cohérent avec un réacteur à eau pressurisée.
  \item La quantité de plutonium consommée par an est probablement d'environ \SI{2,6}{\kilogram}, dérivée de la consommation par TWh et de la production d'énergie électrique annuelle de l'assemblage.
\end{itemize}

\subsection{Nombre de fissions annuelles}
Le calcul du nombre de fissions annuelles dans l'assemblage repose sur la puissance thermique, estimée à environ \SI{15,4}{\mega\watt}, avec une efficacité énergétique typique pour un réacteur à eau pressurisée (REP). En tenant compte de l'énergie libérée par fission (environ \SI{198,2}{\mega\electronvolt}, soit \SI{3,177e-11}{\joule}), et en multipliant par le nombre de secondes dans une année (environ \num{31,5e6}), on obtient environ \SI{1,5e25}{fissions\per\year}. Cette estimation suppose une opération continue, ce qui est une simplification, mais reflète les données fournies.

\subsection{Taux de fission des éléments fissiles}
Pour calculer les taux de fission des principaux éléments fissiles (U-235, Pu-239, Pu-241), nous utilisons des sections efficaces thermiques standard de la littérature : environ \SI{584,4}{\barn} pour U-235, \SI{747}{\barn} pour Pu-239 et \SI{959}{\barn} pour Pu-241. En estimant le flux de neutrons à \SI{3e17}{\neutron\per\centi\meter\squared\per\second} et en tenant compte des densités atomiques calculées pour chaque isotope dans le combustible MOXEUS (un mélange homogène d'uranium enrichi et de plutonium), les taux de fission sont d'environ \SI{1,1e17}{fissions\per\second} pour U-235, \SI{3,1e17}{fissions\per\second} pour Pu-239 et \SI{6,4e16}{fissions\per\second} pour Pu-241. Ces valeurs dépendent de la composition du combustible, ici supposée avec \SI{10}{\percent} de plutonium et \SI{3}{\percent} d'enrichissement en U-235.

\subsection{Flux de neutrons}
L'ordre de grandeur du flux de neutrons, estimé à \SI{1e17}{\neutron\per\centi\meter\squared\per\second}, est cohérent avec les réacteurs à eau pressurisée, où le flux soutient la réaction en chaîne pour maintenir la puissance thermique. Cette estimation est dérivée de la relation entre le taux total de fission et la section efficace macroscopique, en utilisant les données de puissance et de composition du combustible.

\subsection{Quantité de plutonium consommée}
La quantité de plutonium consommée par an, estimée à environ \SI{2,6}{\kilogram}, est calculée en utilisant la consommation de plutonium par TWh (\SI{58}{\kilogram\per\tera\watt\hour} selon les données fournies) et la production annuelle d'énergie électrique de l'assemblage, estimée à environ \SI{0,044}{\tera\watt\hour\per\year}, en tenant compte d'une efficacité thermique de \SI{33}{\percent}. Cette valeur reflète la consommation nette, incluant les fissions de \isotope[239]{Pu} et \isotope[241]{Pu}, mais peut varier en fonction de la composition exacte et des cycles de combustion.