\documentclass{article}
\usepackage[utf8]{inputenc}
\usepackage{amsmath}
\usepackage{geometry}
\geometry{a4paper, margin=1in}
\author{Gabriel Boutry}
\title{Calculs d'Ordres de Grandeur pour le Combustible MOXEUS en REP}
\begin{document}
\maketitle

\section*{Introduction}
\addcontentsline{toc}{section}{Introduction}
La réalisation de calculs d'ordres de grandeur constitue une étape préliminaire essentielle avant d'entreprendre des simulations neutroniques complexes. Cette approche permet non seulement d'anticiper les résultats attendus, mais également de développer un regard critique sur les futures données de simulation. Voici les calculs demandés pour un assemblage de combustible MOXEUS en réacteur à eau pressurisée (REP).

\section{Calcul du nombre de fissions annuelles dans un assemblage}
Pour déterminer le nombre de fissions annuelles, il convient d'abord de calculer la masse totale de combustible présente dans l'assemblage, puis d'établir la relation entre la puissance thermique et l'énergie libérée par fission.

\subsection{Détermination de la masse de combustible dans l'assemblage}
D'après les données fournies, l'assemblage comporte 264 crayons de combustible avec les caractéristiques suivantes :
\begin{itemize}
  \item Rayon de la pastille : 0,410 cm
  \item Hauteur active de l'assemblage : 36,6 cm
  \item Densité du combustible : 10,02 g/cm³
\end{itemize}
Le volume d'une pastille cylindrique par crayon se calcule ainsi :
\begin{equation}
  \text{Volume} = \pi \times r^2 \times h = \pi \times (0,410 \text{ cm})^2 \times 36,6 \text{ cm} = 19,37 \text{ cm}^3
\end{equation}
La masse de combustible par crayon est donc :
\begin{equation}
  \text{Masse} = \text{Volume} \times \text{Densité} = 19,37 \text{ cm}^3 \times 10,02 \text{ g/cm}^3 = 194,09 \text{ g}
\end{equation}
Pour l'ensemble de l'assemblage :
\begin{equation}
  \text{Masse totale} = 194,09 \text{ g} \times 264 \text{ crayons} = 51 239,76 \text{ g} \approx 51,24 \text{ kg}
\end{equation}

\subsection{Calcul de la puissance thermique}
La densité de puissance thermique étant de 30 W/g d'oxyde, la puissance totale de l'assemblage est :
\begin{equation}
  \text{Puissance} = 30 \text{ W/g} \times 51 239,76 \text{ g} = 1 537 192,8 \text{ W} \approx 1,54 \text{ MW}
\end{equation}

\subsection{Détermination du nombre de fissions}
L'énergie libérée par fission est d'environ 200 MeV, soit en joules :
\begin{equation}
  \text{Énergie par fission} = 200 \text{ MeV} \times 1,602 \times 10^{-13} \text{ J/MeV} = 3,204 \times 10^{-11} \text{ J}
\end{equation}
Le nombre de fissions par seconde est alors :
\begin{equation}
  \text{Fissions/seconde} = \frac{1 537 192,8 \text{ W}}{3,204 \times 10^{-11} \text{ J}} = 4,798 \times 10^{16} \text{ fissions/s}
\end{equation}
Sur une année entière :
\begin{equation}
  \text{Fissions/an} = 4,798 \times 10^{16} \times 365,25 \times 24 \times 3600 = 1,514 \times 10^{24} \text{ fissions/an}
\end{equation}
Le nombre de fissions annuelles dans l'assemblage est donc d'environ $1,5 \times 10^{24}$ fissions par an.

\section{Calcul des taux de fission pour les principaux éléments fissiles}
Pour déterminer les taux de fission des différents isotopes, nous devons considérer leurs sections efficaces microscopiques de fission en spectre thermique et leurs densités atomiques respectives dans le combustible.

\subsection{Estimation de la composition isotopique}
Considérons un combustible MOXEUS typique avec :
\begin{itemize}
  \item Une teneur en plutonium de 10\% (dans la plage 0-16\% indiquée)
  \item Un enrichissement en $^{235}$U de 3\% (dans la plage 0,25-5\% indiquée)
  \item Une composition isotopique du plutonium : 50\% de $^{239}$Pu, 30\% de $^{240}$Pu, 10\% de $^{241}$Pu et 10\% d'autres isotopes
\end{itemize}

\subsection{Calcul des densités atomiques}
La densité atomique de chaque isotope peut être calculée par :
\begin{equation}
  N = \frac{\rho \times N_A \times w}{M}
\end{equation}
Où :
\begin{itemize}
  \item $\rho$ est la densité du combustible (10,02 g/cm³)
  \item $N_A$ est le nombre d'Avogadro ($6,022 \times 10^{23}$ atomes/mol)
  \item $w$ est la fraction massique de l'isotope
  \item $M$ est la masse molaire de l'isotope (g/mol)
\end{itemize}
Pour l'uranium-235 (M = 235 g/mol) :
\begin{equation}
  N(^{235}\text{U}) = \frac{10,02 \times 6,022 \times 10^{23} \times 0,03 \times 0,90}{235} = 6,91 \times 10^{20} \text{ atomes/cm}^3
\end{equation}
Pour le plutonium-239 (M = 239 g/mol) :
\begin{equation}
  N(^{239}\text{Pu}) = \frac{10,02 \times 6,022 \times 10^{23} \times 0,10 \times 0,50}{239} = 1,26 \times 10^{21} \text{ atomes/cm}^3
\end{equation}
Pour le plutonium-241 (M = 241 g/mol) :
\begin{equation}
  N(^{241}\text{Pu}) = \frac{10,02 \times 6,022 \times 10^{23} \times 0,10 \times 0,10}{241} = 2,50 \times 10^{20} \text{ atomes/cm}^3
\end{equation}

\subsection{Sections efficaces de fission en spectre thermique}
Les sections efficaces microscopiques de fission pour les principaux isotopes fissiles sont :
\begin{itemize}
  \item $^{235}$U : $\sigma_f \approx 585$ barns = $585 \times 10^{-24}$ cm²
  \item $^{239}$Pu : $\sigma_f \approx 748$ barns = $748 \times 10^{-24}$ cm²
  \item $^{241}$Pu : $\sigma_f \approx 1012$ barns = $1012 \times 10^{-24}$ cm²
\end{itemize}

\subsection{Calcul des taux de fission relatifs}
La contribution de chaque isotope au taux de fission total peut être calculée comme suit :
\begin{equation}
  \text{Contribution de l'U-235} : \frac{6,91 \times 10^{20} \times 585 \times 10^{-24}}{6,91 \times 10^{20} \times 585 \times 10^{-24} + 1,26 \times 10^{21} \times 748 \times 10^{-24} + 2,50 \times 10^{20} \times 1012 \times 10^{-24}} = 25,3\%
\end{equation}
\begin{equation}
  \text{Contribution du Pu-239} : \frac{1,26 \times 10^{21} \times 748 \times 10^{-24}}{\text{même dénominateur}} = 59,0\%
\end{equation}
\begin{equation}
  \text{Contribution du Pu-241} : \frac{2,50 \times 10^{20} \times 1012 \times 10^{-24}}{\text{même dénominateur}} = 15,8\%
\end{equation}
Dans ce combustible MOXEUS, le plutonium contribue donc à environ 75\% des fissions (59\% pour le $^{239}$Pu et 16\% pour le $^{241}$Pu), tandis que l'uranium-235 est responsable d'environ 25\%.

\section{Calcul du flux de neutrons dans l'assemblage}
Le flux neutronique peut être déterminé à partir de la puissance thermique et des taux de réaction de fission des différents isotopes.
La relation entre la puissance, le flux et les sections efficaces de fission est :
\begin{equation}
  \text{Puissance} = \Sigma (N_i \times \sigma_{fi} \times \Phi \times V \times E_f)
\end{equation}
Où :
\begin{itemize}
  \item $N_i$ est la densité atomique de l'isotope $i$
  \item $\sigma_{fi}$ est la section efficace microscopique de fission de l'isotope $i$
  \item $\Phi$ est le flux neutronique
  \item $V$ est le volume du combustible
  \item $E_f$ est l'énergie libérée par fission
\end{itemize}
En réorganisant cette équation pour isoler $\Phi$, on obtient :
\begin{equation}
  \Phi = \frac{\text{Puissance}}{[\Sigma (N_i \times \sigma_{fi}) \times V \times E_f]}
\end{equation}
Avec les valeurs calculées précédemment et un volume de combustible de $5 113,68 cm^3$ ($= 264 \times 19,37 cm^3$), on obtient :
\begin{equation}
  \Phi = \frac{1,54 \times 10^6}{[(6,91 \times 10^{20} \times 585 \times 10^{-24} + 1,26 \times 10^{21} \times 748 \times 10^{-24} + 2,50 \times 10^{20} \times 1012 \times 10^{-24}) \times 5 113,68 \times 3,204 \times 10^{-11}]}
\end{equation}
Ce qui donne :
\begin{equation}
  \Phi \approx 5,88 \times 10^{13} \text{ neutrons/cm}^2/\text{s}
\end{equation}
Cette valeur est tout à fait cohérente avec les flux neutroniques typiquement observés dans les réacteurs à eau pressurisée, qui se situent généralement entre $10^{13}$ et $10^{14}$ n/cm²/s.

\section{Calcul des quantités de Pu consommées par an}
La quantité de plutonium consommée peut être estimée de deux façons différentes.

\subsection{Méthode 1 : À partir des données du tableau 2.5}
D'après le tableau 2.5 fourni dans l'énoncé, la consommation de plutonium pour le concept MOXEUS est de 58 kg/TWh.
Pour notre assemblage d'une puissance de 1,54 MW, la production d'énergie annuelle est :
\begin{equation}
  \text{Énergie annuelle} = 1,54 \text{ MW} \times 365,25 \text{ jours} \times 24 \text{ heures} = 13 493,58 \text{ MWh/an} = 0,013494 \text{ TWh/an}
\end{equation}
La consommation annuelle de plutonium est donc :
\begin{equation}
  \text{Consommation de Pu} = 58 \text{ kg/TWh} \times 0,013494 \text{ TWh/an} = 0,783 \text{ kg/an}
\end{equation}

\subsection{Méthode 2 : Par calcul direct des taux de réaction}
Pour un calcul plus détaillé, nous devons considérer à la fois la disparition du plutonium par fission et par capture neutronique, ainsi que sa production par capture sur l'uranium-238.

\subsubsection{Disparition du plutonium}
En prenant en compte les sections efficaces de capture ($\sigma_c$) en plus des sections efficaces de fission :
\begin{itemize}
  \item $\sigma_c(^{239}\text{Pu}) \approx 270$ barns
  \item $\sigma_c(^{241}\text{Pu}) \approx 360$ barns
\end{itemize}
Le taux de disparition du plutonium est :
\begin{equation}
  \text{Taux} = \Sigma [N_i \times (\sigma_{fi} + \sigma_{ci}) \times \Phi \times V]
\end{equation}
\begin{equation}
  = [(1,26 \times 10^{21} \times (748 + 270) \times 10^{-24}) + (2,50 \times 10^{20} \times (1012 + 360) \times 10^{-24})] \times 5,88 \times 10^{13} \times 5 113,68
\end{equation}
\begin{equation}
  \approx 4,86 \times 10^{17} \text{ atomes/s}
\end{equation}
Ce qui correspond à une masse de :
\begin{equation}
  \text{Masse disparue} = 4,86 \times 10^{17} \times \frac{239 + 241}{2} \times 1,66 \times 10^{-24} = 1,93 \times 10^{-4} \text{ g/s}
\end{equation}
\begin{equation}
  = 1,93 \times 10^{-4} \times 365,25 \times 24 \times 3600 = 6,08 \text{ kg/an}
\end{equation}

\subsubsection{Production de plutonium}
Le plutonium est produit par capture neutronique sur l'uranium-238 :
\begin{itemize}
  \item $\sigma_c(^{238}\text{U}) \approx 2,7$ barns
  \item $N(^{238}\text{U}) \approx 2,26 \times 10^{22} \text{ atomes/cm}^3$ (pour environ 87\% d'U-238 dans le combustible)
\end{itemize}
Le taux de production est :
\begin{equation}
  \text{Taux} = N(^{238}\text{U}) \times \sigma_c(^{238}\text{U}) \times \Phi \times V
\end{equation}
\begin{equation}
  = 2,26 \times 10^{22} \times 2,7 \times 10^{-24} \times 5,88 \times 10^{13} \times 5 113,68
\end{equation}
\begin{equation}
  \approx 1,65 \times 10^{17} \text{ atomes/s}
\end{equation}
Ce qui correspond à une masse de :
\begin{equation}
  \text{Masse produite} = 1,65 \times 10^{17} \times 239 \times 1,66 \times 10^{-24} = 6,54 \times 10^{-5} \text{ g/s}
\end{equation}
\begin{equation}
  = 6,54 \times 10^{-5} \times 365,25 \times 24 \times 3600 = 2,06 \text{ kg/an}
\end{equation}

\subsubsection{Bilan net de consommation}
La consommation nette de plutonium est donc :
\begin{equation}
  \text{Consommation nette} = 6,08 - 2,06 = 4,02 \text{ kg/an}
\end{equation}
Il existe un écart entre cette valeur calculée (4,02 kg/an) et celle déduite du tableau 2.5 (0,783 kg/an). Cette différence peut s'expliquer par plusieurs facteurs, notamment les hypothèses simplificatrices sur la composition isotopique du combustible, le spectre neutronique considéré comme purement thermique dans nos calculs, et les incertitudes sur les sections efficaces utilisées.

\section{Conclusion}
Les calculs d'ordres de grandeur réalisés nous permettent d'établir les valeurs suivantes pour un assemblage de combustible MOXEUS en REP :
\begin{itemize}
  \item Nombre de fissions annuelles : environ $1,5 \times 10^{24}$ fissions/an
  \item Taux de fission : 25\% pour l'U-235, 59\% pour le Pu-239 et 16\% pour le Pu-241
  \item Flux neutronique : environ $5,9 \times 10^{13}$ neutrons/cm²/s
  \item Consommation nette de plutonium : entre 0,8 et 4 kg/an, selon la méthode de calcul employée
\end{itemize}
Ces valeurs constituent une base de référence pour l'interprétation des résultats des simulations neutroniques qui seront effectuées ultérieurement.

\end{document}